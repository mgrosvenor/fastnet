%%%%%%%%%%%%%%%%%%%%%%%%%%%%%%%%%%%%%%%%%%%%%%%%%%%%%%%%%%%%%%%%%%%%%%%%%%%
% 2) Experimental Setup
%%%%%%%%%%%%%%%%%%%%%%%%%%%%%%%%%%%%%%%%%%%%%%%%%%%%%%%%%%%%%%%%%%%%%%%%%%%

%Making a conscious effort to remove "britishisms" from the text -- thus, hence, whilst, while, yet, furthermore, whereas, etc. 
%Also trying to slim down on adjectives. One mans's "very" is another man's "some". 
%To make diff/merge easier. I am now putting 1 "statement" (sentence) per line. 
%Trying to apply the KISS principle, especially to the language used, to help out our American reviewers. 

\begin{algorithm}

 \SetKw{To}{to}
 \SetKw{StartTimer}{start $\leftarrow$ timer}
 \SetKw{StopTimer}{stop $\leftarrow$ timer}
 \SetKw{Continue}{continue}
 \ForEach{ packet in packet\_buffer }{
    \BlankLine          
     \StartTimer{}\;
     \BlankLine
     packet.ethernet.src $\leftarrow$ 0xFFFFFF\; 
     packet.ethernet.dst $\leftarrow$ 0x000001\;
     packet.ethernet.type $\leftarrow$ 0x0800\; 
     $\cdots$\\     
     packet.ip.size $\leftarrow$ \emph{big\_endian}(IP\_SIZE)\;
     $\cdots$\\     
     packet.ip.protocol $\leftarrow$  0x11\;
     $\cdots$\\         
     packet.udp.size $\leftarrow$ \emph{big\_endian}(UDP\_SIZE)\;          
     $\cdots$\\
     \For{ i $\leftarrow$ 0 \To{} UDP\_SIZE $\div{}$ WORD\_SIZE}{
         packet.udp.data[\emph{i}] $\leftarrow$ const integer\;
     }
     \BlankLine
     \StopTimer{}\;
     \BlankLine
 }


\caption{Generating packets}
\label{alg:net_gen}
\end{algorithm}


\section{Testing High Speed Networks}
\label{s:experiments}
To test the effect of packet layouts on processing speed, we would like to use a flexible, high speed network adapter. 
Some FPGA based 100Gb/s network adapters are beginning to appear on the market~\cite{100gnic}, but they are rare and expensive. 
Additionally, adapting 1000's of lines of legacy network stack code to process new packet formats will be time consuming. 
To work around these problems, we take a different approach. 
Instead of building a network stack for a particular adapter, we use our experience in high speed network adapter design \emph{[reference removed for blind review]} to build a generic network stack that runs directly out of system memory. 
We assume that an adapter will eventually become available to place packets there. 
This allows us to test the absolute speed of packet processing quickly and flexibly 
It also gives us an indication of the expected upper limits that software could expect to achieve. 
By writing our own network stack, we free ourselves from legacy code bases and can ensure that our packet processor is flexible enough to cope with a range of different packet formats. 


Our initial network stack only processes UDP packets, delivered over IP version 4.0 transport using Ethernet frames. 
In principle any protocol or transport could be supported, but this is the simplest to implement. 
By doing so, we can get a strong indication of the practical upper bounds of any reasonable network stack. 


Our network stack is divided into two parts: generating and receiving packets. 
Packet generation is detailed in Algorithm~\ref{alg:net_gen} and packet receiving is detailed in Algorithm~\ref{alg:net_rcv}. 
For each experiment, we first run the generation phase and then the receiving phase. This way we can test both parts independently. 
As with any sensible network stack, we only perform endian conversions for values that are non-constant. 
For example, the ethernet packet type field is a constant, but the IP header size is not. 
Our packet buffer is pinned so that swapping cannot happen and aligned to a 4kB page boundary. 

Our experiments are run using a top-of-the range Intel Core i7 Ivy Bridge CPU (4930K), with 12 threads running at 3.4GHz. 
We ensure that the network-stack process is pinned to a unique CPU, with no other processes or hyperthreads interfering and that the process is run with full-realtime priority. 
This environment is overly generous. 
Real in-kernel network stacks have limited time to run and have to contend with interrupts and threads from other competing resources. 




\begin{algorithm}
 \SetKw{Continue}{continue}
 \SetKw{To}{to}
 \SetKw{StartTimer}{start $\leftarrow$ timer}
 \SetKw{StopTimer}{stop $\leftarrow$ timer}

 \ForEach{ packet in packet\_buffer }{
     \BlankLine
     \StartTimer{}\;
     \BlankLine
     \If{ packet.ethernet.src $\neq$ 0xFFFFFF }{
        \Continue{};
     }
     \If{ packet.ethernet.dst $\neq$ 0x000001 }{
        \Continue{};
     }
     \If{ packet.ethernet.type $\neq$  0x0800 }{
        \Continue{};
     }
     $\cdots$\\
     ip\_size $\leftarrow$ \emph{little\_endian}(packet.ip.size)\;
     total\_ip\_bytes $\leftarrow$ total\_ip\_bytes + ip\_size\;
     $\cdots$\\     
      \If{ packet.ip.protocol $\neq$  0x11 }{
        \Continue{}\;
     }
     $\cdots$\\         
     udp\_size $\leftarrow$ \emph{little\_endian}(packet.udp.size)\;     
     total\_udp\_bytes $\leftarrow$ total\_udp\_bytes + udp\_size\;     
     $\cdots$\\
     counter $\leftarrow$ 0\;
     \For{ i $\leftarrow$ 0 \To{} udp\_size $\div{}$ WORD\_SIZE}{
         counter $\leftarrow$ packet.udp.data[\emph{i}];
     }
     \BlankLine
     \StopTimer{}\;    
     \BlankLine
     
 }

\caption{Receiving packets}
\label{alg:net_rcv}
\end{algorithm}

